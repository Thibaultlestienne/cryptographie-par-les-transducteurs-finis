
\usepackage{enumitem} % Pour personnaliser les listes
\usepackage[T1]{fontenc}
\usepackage{amsmath, amssymb} % Pour les mathématiques
\usepackage{xcolor} % Pour les couleurs
\usepackage{geometry} % Pour ajuster les marges
\usepackage{tikz}
\usetikzlibrary{automata, positioning}

% Configuration des marges
\geometry{
    a4paper,
    left=2cm,
    right=2cm,
    top=2cm,
    bottom=2cm
}

% Définition des couleurs
\definecolor{bleuclair}{rgb}{0.8, 0.9, 1}
\definecolor{rougeclair}{rgb}{1, 0.8, 0.8}
\definecolor{vertclair}{rgb}{0.8, 1, 0.8}
\definecolor{grisclair}{rgb}{0.9, 0.9, 0.9}
\definecolor{nouvellecouleur}{rgb}{1, 1, 0.7}

% Compteurs pour les sections numérotées
\newcounter{numdefinition}
\newcounter{numtheoreme}
\newcounter{numdemonstration}
\newcounter{numexemple}
\newcounter{numremarque}
\setcounter{numdefinition}{0}
\setcounter{numtheoreme}{0}
\setcounter{numdemonstration}{0}
\setcounter{numexemple}{0}
\setcounter{numremarque}{0}

% Commande pour les définitions
\newcommand{\definition}[2]{%
    \vspace{1em}%
    \stepcounter{numdefinition}%
    \noindent%
    \fcolorbox{black}{bleuclair}{%
        \parbox[t]{0.95\textwidth}{%
            \textbf{\sffamily Définition \thenumdefinition : #1} \par % Nom de la définition
            #2 % Contenu de la définition
        }%
    }%
    \vspace{1em}%
}

% Commande pour les théorèmes
\newcommand{\theoreme}[2]{%
    \vspace{1em}%
    \stepcounter{numtheoreme}%
    \noindent%
    \fcolorbox{black}{rougeclair}{%
        \parbox[t]{0.95\textwidth}{%
            \textbf{\sffamily Théorème \thenumtheoreme : #1} \par % Nom du théorème
            #2 % Contenu du théorème
        }%
    }%
    \vspace{1em}%
}

% Commande pour les démonstrations
\newcommand{\demonstration}[1]{%
    \vspace{1em}%
    \stepcounter{numdemonstration}%
    \noindent%
    \fcolorbox{black}{vertclair}{%
        \parbox[t]{0.95\textwidth}{%
            \textbf{\sffamily Démonstration \thenumdemonstration :} \par%
            #1%
        }%
    }%
    \vspace{1em}%
}

% Commande pour les exemples
\newcommand{\exemple}[2]{%
    \vspace{1em}%
    \stepcounter{numexemple}%
    \noindent%
    \fcolorbox{black}{grisclair}{%
        \parbox[t]{0.95\textwidth}{%
            \textbf{\sffamily Exemple \thenumexemple : #1} \par%
            #2%
        }%
    }%
    \vspace{1em}%
}


% Commande pour les remarques
\newcommand{\remarque}[2]{%
    \vspace{1em}%
    \stepcounter{numremarque}%
    \noindent%
    \fcolorbox{black}{nouvellecouleur}{%
        \parbox[t]{0.95\textwidth}{%
            \textbf{\sffamily Remarque \thenumremarque : #1} \par%
            #2%
        }%
    }%
    \vspace{1em}%
}
