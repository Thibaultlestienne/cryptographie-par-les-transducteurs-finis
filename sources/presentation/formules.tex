\documentclass[a4paper,10pt]{article}
\usepackage[utf8]{inputenc}

%opening
\title{TIPE}
\author{T.Lestienne}

\begin{document}

\maketitle

\section{chiffre}

\noindent
\textbf{Definition 1.1} \\
Un \textit{chiffre} defini sur $(\mathcal{K}, \mathcal{M}, \mathcal{C})$, avec $\mathcal{K}$ le  l'\textit{espace des clés}, $\mathcal{M}$ l' \textit{espace des messages}, et $\mathcal{C}$ l' \textit{espace des messages encrypter}, est $(E, D)$
\begin{itemize}
    \item $E : \mathcal{K} \times \mathcal{M} \to \mathcal{C}$ is the encoder,
    \item $D : \mathcal{K} \times \mathcal{C} \to \mathcal{M}$ is the decoder,
    \item $\forall m \in \mathcal{M}, \, \forall k \in \mathcal{K}, \, D(k, E(k, m)) = m$.
\end{itemize}

\noindent
Observe that we want to have the encoding and decoding functions $E, D$ to be efficient, because in practice, $\mathcal{K}$ is 128 or 256 bits (may be bigger) and $\mathcal{M}, \mathcal{C}$ are 1 GByte big (a whole file).


\end{document}
